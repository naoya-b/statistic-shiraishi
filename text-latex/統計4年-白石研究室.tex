\documentclass[a4j,12pt]{jarticle}
\usepackage{amsmath}
\usepackage{amssymb}
\usepackage{amsfonts}
\usepackage{enumitem}
\begin{document}

\title{Ruppret・Matteson-Statistics}
\author{伴 尚哉\thanks{\texttt{naoyayg30@keio.jp}}\\61715382\\慶應義塾大学理工学部数理科学科 白石研究室}
\date{\today}
\maketitle

\section{Introduction}
\ 略
\section{利益率}
\subsection{導入}
\ 投資の目的はもちろん利益を出すことである。投資で利益を得るかどうかは、自分の資産の量と投資の値段に依存する。投資家は、収入に大変興味があり、それは初期投資に大きく関わる。
利益率は、初期値段の何分の1かといった形で表記される。
\subsubsection{Net Return(純利益率)}
\ $P_{t}$を時間$t$での資産額とする。配当がないと仮定すると、$t-1 ~ t$時間までの純利益率は以下の式で表せる。
\begin{align*}
R_{t} = \frac{P_{t}}{P_{t-1}}-1 = \frac{P_{t} - P_{t-1}}{P_{t-1}}
\end{align*}
ここで、分子の$P_{t} - P_{t-1}$は、期間内の利益や収入を表す。もし、値が負なら資産を失ったことを表す。分母の$P_{t-1}$は期間内での初期の投資額を表す。それゆえ、純利益率というのは、相対的な収入・利益率等と見ることもできる。$\\ $
\ 試算を持つことによる利益率というのは以下の式で表せられる。
\begin{align}
revenue = initial investment × net return
\end{align}
\ 例えば、初期資産が$\$10000$で、純利益率が$6\%$としたら、$\$600$の収入を得ることになる。なぜなら、$P_{t} \leq 0 $だからである。また、$R_{t} \leq -1$である。起こりえる最も悪いリターン値は、-1の時、つまり$100\%$失うことであり、資産の価値がなくなるときにおこる。

\subsubsection{Gross Returns(総利益率)}
\ 単純な総利益率は以下の式で定義できる。
\begin{align}
\frac{P_{t}}{P_{t-1}} = 1 + R_{t}
\end{align}
\ 例えば、$P_{t}=2 \; and \; p_{t+1} = 2.1$ならば、$1 + R_{t+1} = 1.05 (= 105\%)$となり、$R_{t+1} = 0.05 (5\%)$となる。時点tでの最終的な資産は、時点t-1での初期資産に総利益率(Gross Return)をかけたものとなる。
別の方法で表記すると、もし、時点t-1での初期資産が$X_{0}$ならば、時点tでの資産は$X_{0}(1 + R_{t})$となる。$\\ $
\ 利益率というのは、単位がない - つまり、ドルやセントといった単位に依存しないのである。しかし、利益率に単位が必要ないわけではない。利益率の単位とは時間である。つまり、時間、曜日といったものに依存するのだ。
例えば、時間tが年単位で計測されているならば、より正確に書くと純利益率は$5\% / year$となる。 $\\ $
\ 直近$k$期間での総利益率のことを、k-期間ごとの総利益率(つまり、時点t-k~tまでの期間)と呼び、以下で定義される。
\begin{align}
1 + R_{t}(k) = \frac{P_{t}}{P_{t-k}} &= \left(\frac{P_{t}}{P_{t-1}}\right)\left(\frac{P_{t-1}}{P_{t-2}}\right)\cdots\left(\frac{P_{t-k+1}}{P_{t-k}}\right) \\ \nonumber
                                     &= (1+R_{t})\cdots(1 + R_{t-k+1})
\end{align}
ただし、時点kでの純利益率を$R_{k}$と表す。

\subsubsection{log Returns(対数利益率)}
\ 対数利益率は、連続的に組み込まれる利益とも呼ばれ、$r_{t}$で表記される。定義は以下のとおりである。
\begin{align}
r_{t} = \log(1 + R_{t}) = \log \left(\frac{P_{t}}{P_{t-1}}\right) = p_{t} - p_{t-1}
\end{align}
ただし、$p_{t} = \log(P_{t})$であり、対数資産と呼ばれる。$\\ $
\ 対数資産は、大まかには利益率と同じだとみることができる。なぜなら、$x$が小さい時、$\log(1+x) \approx x$と近似できるからである(本書の図(2.1)参考)。例えば利益率が$10\%$以下のとき、
つまり$|x| < 0.1$のとき、図を見れば$\log(1+x)$が、$x$にとても近づくことがわかる。$\\ $
\ 例えば、$5\%$の利益率というのは、$4.88\%$の対数利益率と同じである。なぜなら、$\log(1+0.05) = 0.0488$だからだ。
同様に、$-5\%$の利益率というのは、$-5.13\%$の対数利益率と同じである。なぜなら、$\log(1-0.05) = -0.0513$だからだ。どちらの例でも、$r_{t} = \log(1 + R_{t}) \approx R_{t}$が言える。
もう一つの例は略すが、対数利益というのが純利益と大変近くなることがわかる。利益率は、短期間では対数利益率とほぼ等しくなるくらい小さくなることから、
年利益率と同様、日利益率も対数利益率と利益率が等しいとみることができる一方、10年といった長い期間では、二つを等しく扱う必要はない。$\\ $
\ 利益率と対数利益率は同じ正負の符号を持つ。対数利益率は、正の時は利益率より小さく、負の時は利益率より大きく出る。両者の違いは、どちらの値も負の値に大きかった時に顕著になる。例えば、利益率が$-1$のとき、対数利益率は$-\infty$となる。このとき、すべての資産を失っている。 $\\ $
\ 対数利益率をつかう一つのメリットは、複数の期間にまたがったとき簡単になることである。k期間での対数利益率は、1期間での対数利益率の和で書ける。よって、k期間での対数利益率は
\begin{align}
r_{t}(k) &= \log \{1 + R_{t}(k)\}  \nonumber \\
         &= \log \{(1+R_{t}) \cdots (1+R_{t-k+1})\} \nonumber \\
         &= \log(1 + R_{t}) + \cdots + \log(1 + R_{t-k+1}) \nonumber \\
         &= r_{t} + r_{t-1} + \cdots + r_{t-k+1} \nonumber
\end{align}
\newpage

\subsection{配当の配布}
\ 多くの大企業の株では、利益率を計算したのを考慮にいれ配当を配る。同様に、金利も払う。もし配当や金利$D_{t}$が時点$t$の前に払われたとする。その時、時点$t$での総利益は以下のように定義できる。
\begin{align}
1 + R_{t} = \frac{P_{t}+D_{t}}{P_{t-1}}
\end{align}
よって、純利益は$R_{t} = (P_{t} + D_{t})/P_{t-1} - 1$となる。そして、対数利益率は、$r_{t} = \log(1+R_{t}) = \log(P_{t} + D_{t}) - \log(P_{t-1})$となる。
複数期間にまたがる総利益率は、短期間の利益率の集まりでかけ、つまり、
\begin{align}
1 + R_{t}(k) &= \left(\frac{P_{t} + D_{t}}{P_{t-1}}\right)\left(\frac{P_{t-1} + D_{t-1}}{P_{t-2}}\right)\cdots\left(\frac{P_{t-k+1} + D_{t-k+1}}{P_{t-k}}\right) \nonumber \\
&=(1+R_{t})(1+R_{t-1})\cdots(1+R_{t-k+1})
\end{align}
もし、期間$s-1 \sim s$で配当がなかったなら、ある時点$s$で$D_{s} = 0$となる。
同様に、期間$k$での対数利益率は、
\begin{align*}
r_{t}(k) &= \log(1+R_{t}(k)) = \log(1+R(t)) + \cdots + \log(1+R(t-k+1)) \nonumber \\
&=\log\left(\frac{P_{t} + D_{t}}{P_{t-1}}\right)+\cdots+\log\left(\frac{P_{t-k+1} + D_{t-k+1}}{P_{t-k}}\right)
\end{align*}

\subsection{ランダムウォークモデル}
\ ランダムウォークという仮説は、期間ごとの対数利益率$r_{t} = \log(1 + R_{t})$は独立であると述べている。なぜなら、
\begin{align*}
1 + R_{t}(k) &= (1+R_{t})\cdots(1+R_{t-k+1}) \\
&= \exp(r_{t}) \cdots \exp(r_{t-k+1}) \\
&= \exp(r_{t}\cdots r_{t-k+1})
\end{align*}
であり、
\begin{align}
\label{b}
\log\{1+R_{t}(k)\} = r_{t} + \cdots + r_{t-k+1}
\end{align}
となるからである。$\\ $
\ よく、対数利益率は一定の平均・分散の正規分布であると考えられることが多い。正規分布の再生成から、単期間の対数利益率が複数機関の対数利益率を表す。この仮定の下では、
$\log\{1+R_{t}(k)\}$は$N(k\mu,k\sigma^{2})$に従う。
\newpage
\subsubsection{ランダムウォーク}
\ 式(\ref{b})は、ランダムウォークモデルの一つの例である。今、$Z_{1},Z_{2},\ldots$が平均$\mu$、分散$\sigma$に独立同一に従うとする。このとき、$S_{0}$を任意の始点とする。
\begin{align}
\label{c}
\S_{t} = S_{0} + Z_{1} + \cdots + Z_{t},\; t\geq 1.
\end{align}
とすると、式(\ref{c})より、$S_{t}$は$t$ステップ後の位置となる。$\\ $
\ $S_{0},S_{1},\ldots$という過程は、ランダムウォークと呼ばれ、$Z_{1},Z_{2}\ldots$を過程と呼ぶ。もしこの仮定が、標準正規分布なら。標準ランダムウォークと呼ばれる。
$S_{0}$が与えられた下での$S_{t}$の期待値と分散はそれぞれ$E(S_{t}|S_{0}) = S_{0} + \mu t$と$Var(S_{t}|S_{0}) = \sigma^{2}t$となる。
このとき、パラメータ$\mu$は傾向とよばれ、ランダムウォークの動く方向を決める。パラメータ$\sigma$は変動と呼ばれ、$S_{t}$の条件付き期待値からどれほど値が広がるかというのを表している。
$S_{0}$が与えられて時の$S_{t}$が$68\%$の確率で含まれる位置は$(S_{0} + \mu t) \pm \sigma \sqrt{t}$と書ける。$68\%$範囲は、$\sqrt{t}$が増えるごとに大きくなっていく。
図2.2を見ると、ランダムウォークの位置の$t=0$の時の予測は、少し時間がたった後の予測より狭いことがわかる。
\subsection{幾何ランダムクォーク}
\ $\log \{ 1 + R_t{k}\} = r_{t} + \cdots + r_{t-k+1}$であることを使うと、以下が成り立つ。
\begin{align}
\frac{P_{t}}{P_{t-k}} = 1 + R_{t}(k) = exp(r_{t} + \cdots + r_{t-k+1})
\end{align}
よって、$k=t$ととると、
\begin{align}
P_{t} = P_{0}exp(r_{t} + r_{t-1} + \cdots + r_{1})
\end{align}
が成り立つ。この過程のことを幾何ランダムウォークもしくは$\exp$ランダムウォークと呼ぶ。もし、$r_{i}$が独立同一に$N(\mu,\sigma^2)$に従う場合、$P_{t}$はすべての時点で対数正規分布となる。
このとき、パラメータ$(\mu,\sigma^{2})$の対数正規・幾何ランダムウォークと呼ばれる。A9.4は略。また、$mu$も対数正規・幾何ランダムウォークの対数傾向と呼ばれる。

\subsubsection{対数価格は、対数正規幾何ランダムウォークに従うか?}
\ 金融数理の多くの仕事では、価格は対数正規幾何ランダムウォークか、連続時間でのブラウン運動に従うと考えられている。なので、自然な疑問として、この仮定が一般的に正しいのかどうかというのがある。
そして、その答えはノーである。対数正規・幾何ランダムウォークは二つの仮定がある。一つ目は、対数利益率は離散値であること・二つ目は、対数利益率は独立であることである。$\\ $
\ 4章・5章では、複数機関における対数利益率の周辺分布について述べる。結論としては、対数利益の密度関数のほうが利益の密度関数より鈴方をしている一方で、裾は対数利益のほうが広いのである。
典型的には、パラメータの自由度4~6の小さい時のt分布は、一般的な利益のモデルより当てはまりが良い。しかし、対数利益分布ではt分布とほぼ同型である。
\ 独立性の仮定も同様に厳しいものである。まず第一に、利益率にはいくつかの相関がある。しかし、一般には、この相関は極めて小さいものである。もっと深刻なものとして、volality-clusteringがある。
これは、大きな変化の後にはしばらく大きな変化が来やすいということだ。このvolality-clusteringは、平方利益率を見れば関係をつかめやすい。$\\ $
\ しかし、章の頭の仮定を切って捨てる前に、Boxさんの発言を思い出すのが大事だ。つまり、「一つのモデルとして正しいものはないが、便利なものはいくつかある」のだ。先の仮定は、有名な"ブラックショールズ方程式"を導出するのに役に立つ。
\subsection{参考文献}
略
\subsection{R}
略
\section{Fixed Income Securitied(確定利子付証券)}
\subsection{導入}
\ 企業は株や債券を売ることで会計を成り立たせている。公開された株を持つことは、その企業の一部のオーナーであるということだ。株主は、企業の利益も損失も一緒に請け負うのだ。
一方、債券は少し違う。もしあなたが債券を買ったなら、企業にローンを貸していることになる。しかしローンと違う点は、債券は売買ができるところである。企業は元本を返金するだけでなく、
債券に紐づけられた利息も払う必要がある。また、債券保持者は、利息ではなく確定利子を受け取ることもできる。これらの理由から、債券は、"確定利子付証券"とも呼ばれる。$\\ $
\ 債券は安全で変動がすくないと思われていてるが、以下はそうでない場合だ。多くの債券は5,10,20年と長い期間である。もし、破産しない企業の債券を買ったり、アメリカの国債を買えば、
あなたが期限まで債券を持っている限りあなたの債券からの収入は保証される。もし、期限前に売ってしまうと、その時の債券の値段に依存した額しかもらうことができない。
債券の値段は金利と逆の方向に動く。つまり、金利が低くなると債券の値段は上がるのだ。長期間の債券はこの影響をより受けやすい。あなたの債券の利率は変わらないものの、債権全体の利率は変動する。
つまり、あなたが持っている債券の値段も変動するのだ。例えば、あなたが$5\%$の金利が払われる債券をかったとして、金利が$6\%$に上がったとする。すると、あなたの債券は新しい債券より低条件である。
結果として、あなたの債券の価値は減少する。ここで売れば、あなたの損失となるのだ。$\\ $
\ 債券の金利はその持ってる期間に依存する。例えば、3か月持てば$4\%$なのが、2年なら$5\%$といったかたちである。このように、どうやって金利が変化していくかを、この章では述べる。
\subsection{Zero-Coupon Bonds(利息がない債券)}
\ 利息がない債券とは、純割引積や"zeros"として呼ばれる金利や特典が満期まで払われない債券のことである。この債券は、満期に払われる表額や譜面価格があり、その表額より安い値段で売られている。
なぜなら、ゼロ割積だからである。$\\ $
\ 例えば、満期20年のゼロ割積で表額が$1000$ドルで年間$6\%$利率が含まれていくとする。つまり、この債券の現在価格は
\begin{align*}
\frac{1000}{(1.06)^{20}} = 311.8
\end{align*}
となる。年利は$6\%$だが、半期ごとに組み込まれている場合、
\begin{align*}
\frac{1000}{(1.03)^{20}} = 306.56
\end{align*}
となる。もし連続時間ならば
\begin{align*}
\frac{1000}{\exp((0.06)^{20})} = 301.19
\end{align*}
となる。
\subsubsection{値段や利益は金利によって変動する}
\ 具体的に、半年ごとに組み込まれる場合について考える。先の例のゼロ割債を$305.56$ドルで買ったとし、半年後に金利が$7\%$に上がったとする。このとき、債券の現価は
\begin{align*}
\frac{1000}{(1.035)^{39}} = 261.41
\end{align*}
となる。つまり、あなたの債権の価値は$45$ドル近く落ちたのだ。$20$年もてば$1000$ドル払われるが、今売ってしまうと$45$ドル損してしまう。半期で$-45/306 = -14.8\%$の損である。
年間では$29\%$近い損だ。たった金利が$1\%$あがっただけなのにだ。つまり、金利が上がると債券の値段が下がるのだ。逆に$1\%$下がれば、年で$19\%$利益があることになり、ステイなら年$6\%$の利益のままである。しかし、たとえ金利が上がろうと下がろうと$20$年持ち続ければ$1000$ドルという金額は保証されているのだ。(本書からかなり省略)
\subsubsection{一般式}
\ ゼロ割債券の値段は以下で与えられる。
\begin{align*}
PRICE = PAR(1 + r)^{-T}
\end{align*}
となり、もし年率$r$で半期ごとに組み込まれるなら
\begin{align*}
PRICE = PAR(1 + r/2)^{-T/2}
\end{align*}
となる。
\subsection{利息付債権}
\ 利息付債権とは、定期的に金利が払われる債券である。この債券は大体譜面価格かそれに近い価格で売られる。満期になれば、債券の譜面通りの額と最後の金利が払われる。例を一つ出す。
譜面額1000ドルで満期20年、年利が6$\%$で半期ごとに払われる債券を考える。つまり、$3\%$が年利に組み込まれる。各年で30ドル配られるとする。債券保持者が40年間この金利を受け取るとすると、
20年後に1000ドルが帰ってくる。これは以下の式からもわかる。
\begin{align*}
\sum_{t=1}^{40}\frac{30}{(1.03)^{t}} + \frac{1000}{(1.03)^40} = 1000
\end{align*}
6か月後には、もし金利が変わらなければ、債券の価値は以下のようになっている。
\begin{align*}
\sum_{t=1}^{30}\frac{30}{(1.03)^{t}} + \frac{1000}{(1.03)^39} = 1030
\end{align*}
これで、期待通り年利6$\%$の半年分の利益が含まれていることがわかる。もし、金利が途中で$7\%$に変わったとき、半年後の価値は
\begin{align*}
\sum_{t=1}^{30}\frac{30}{(1.035)^{t}} + \frac{1000}{(1.035)^39} = (1.035)\left(sum_{t=1}^{40}\frac{30}{(1.035)^{t}} + \frac{1000}{(1.035)^40}\right) = 924.49
\end{align*}
つまり、年間の利率は
\begin{align*}
2\left(\frac{924.49 - 1000}{1000}\right) = -15.1\%
\end{align*}
となっている。もし、金利が$5\%$に落ちたなら、半年後には同様の計算方法で$1153.70$ドルとなり、年利は$30.72\%$となる。
\subsubsection{一般式}
ここでは、よく使われる式を導出する。譜面額PARで、満期T年、半期ごとにCが払われ、半期ごとの金利がrの債券があるとする。このとき、債券の価値は
\begin{align}
\label{d}
\sum_{t = 1}^{2T}\frac{C}{(1+r)^t} + \frac{PAR}{(1+r)^2T} &= \frac{C}{r}\{1-(1+r)^{-2T}\} + \frac{PAR}{(1+r)^2T} \nonumber \\
&=  \frac{C}{r} + \{PAR - \frac{C}{r}\}(1+r)^{-2T}
\end{align}
と書ける。式(\ref{d})から、以下がわかる。時系列モデルでの和の公式で以下がある。$r \neq 1$のとき
\begin{align}
\sum_{i = 0}{T}r^{i} = \frac{1 - r^{T+1}}{1-r}
\end{align}
となる。よって、
\begin{align}
\sum_{t = 1}^{2T}\frac{C}{(1+r)^t} &= \frac{C}{(1+r)}\sum_{t = 1}^{2T-1}\left(\frac{1}{(1+r)}\right)^{t} = \frac{C\{1 - (1+r)^{-2T}\}}{(1+r)\{1 - (1+r)^{-1}\}} \nonumber \\
&= \frac{C}{r}\{1 - (1+r)^{-2T}\}
\end{align}
\end{document}
